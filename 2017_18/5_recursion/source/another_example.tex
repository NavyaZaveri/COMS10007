%another_example.tex
%notes for the course algorithms COMS10007 taught at the University of Bristol
%Conor Houghton conor.houghton@bristol.ac.uk

%To the extent possible under law, the author has dedicated all copyright 
%and related and neighboring rights to these notes to the public domain 
%worldwide. These notes are distributed without any warranty. 


\documentclass[11pt,a4paper]{scrartcl}
\typearea{12}
\usepackage{graphicx}
\usepackage{listings}
\lstset{language=C}
\usepackage{fancyhdr}
\pagestyle{fancy}
\lfoot{\texttt{github.com/conorhoughton/COMS10007}}
\lhead{COMS10007 - algorithms 5\_recursion (e/g) another example - Conor}
\begin{document}

\subsection*{5 Recursion - another example}

Here is another tricky example, this appeared in a worksheet in this course in the past but it was left out this year; I include it for interest only.

Consider $T(n)=T(n-1)+3n$ with $T(1)=1$. Now telescoping gives bits
that look like $3n+3(n-1)+3(n-2)+\ldots$, in other words, you seem to
get an $n$ in every iteration of the telescope. Since there are $n$
iterations of the telescope you might guess
\begin{equation}
T(n)=An^2+Bn+C
\end{equation}
so substituting that in gives
\begin{equation}
An^2+Bn+C=A(n-1)^2+B(n-1)+C+3n=An^2-2An+A+Bn-B+C+3n
\end{equation}
or, after cancelling
\begin{equation}
-2An+A-B+3n=0
\end{equation}
so $A=3/2$ and $B=A$. Thus
\begin{equation}
T(n)=\frac{3}{2}n^2-\frac{3}{2}n+C
\end{equation}
and the initial condition means $C=1$.

\end{document}
