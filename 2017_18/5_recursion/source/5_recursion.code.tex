\documentclass{beamer}
\usepackage{listings}
\lstset{language=C}

\begin{document}

\subsection*{5 Recursion}
\begin{frame}[fragile]{factorial}
\begin{lstlisting}[numbers=left]
int factorial(int n)
{
   if(n<2)
      return 1;

   return n*factorial(n-1);
}
\end{lstlisting}
\end{frame}


\begin{frame}[fragile]{factorial with ternary operator}
\begin{lstlisting}[numbers=left]
int factorial(int n)
{
   return (n<2) ? 1 : n*factorial(n-1);
}
\end{lstlisting}
\end{frame}


\begin{frame}[fragile]{ternary ans = a \& b : c }
\begin{lstlisting}[numbers=left]
if (a)
   ans=b;
else 
   ans=c;
\end{lstlisting}
\end{frame}

%% \begin{frame}[fragile]{factorial with tail recursion}
%% \begin{lstlisting}[numbers=left]
%% int factorial_r(int n, int big_n)
%% {
%%   if(n<2)
%%     return big_n;

%%   return factorial_r(n-1,n*big_n);
%% }

%% int factorial(int n)
%% {
%%   return factorial_r(n,1);
%% }
%% \end{lstlisting}
%% \end{frame}

\begin{frame}[fragile]{binary search with recursion 1}
\begin{lstlisting}[numbers=left]
int search(int a[],int n, int val)
{
  return find_r(a,val,0,n-1);
}
\end{lstlisting}
\end{frame}

\begin{frame} [fragile]{binary search with recursion 2}
\begin{lstlisting}[numbers=left,firstnumber=5]
int find_r(int a[],int val,int low,int high)
{

  if(high<low)
    return -1;

  int mid=(high+low)/2;

  if(a[mid]==val)
    return mid;

  if(val>a[mid])
    return find_r(a,val,mid+1,high);
  
  return find_r(a,val,low,mid-1);
}
\end{lstlisting}
\end{frame}
\end{document}
