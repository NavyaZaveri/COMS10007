%counting_challenge.tex
%problem set for the course COMS10007 taught at the University of Bristol
%Conor Houghton conor.houghton@bristol.ac.uk

%To the extent possible under law, the author has dedicated all copyright 
%and related and neighboring rights to these notes to the public domain 
%worldwide. These notes are distributed without any warranty. 

\documentclass[11pt,a4paper]{scrartcl}
\typearea{12}
\usepackage{graphicx}
\usepackage{pstricks}
\usepackage{listings}
\usepackage{color}

\newif\ifanswers
\answerstrue

\lstset{language=C}
\pagestyle{headings}
\markright{COMS10007 - algorithms fun question - Conor}
\begin{document}

\subsection*{Fun question}

This question is intended to help you practice counting things. In the
Arthur C. Clarke short story \textsl{The Nine Billion Names of God} a
\lq{}Mark V Automatic Sequence Computer\rq{} is purchased by a
religious community to list all the possible names of their deity, a
task whose importance is central to their belief system. We are told
that this name is nine or fewer characters long and that no characters
can occur more than three times in sequence, so AABBAABBA is a
possible name, as is AAABBAABB, but AAAABBAAB is not (see note
below). We are not told in the story how many characters the alphabet
contains but for definiteness lets assume there are 18 letters in the
alphabet. Show how you could calculated how many names there are with
$n$; in other words, describe a method for calculating the number of
names of up to length $n$ and use this method to calculate the
number of names of length nine.

\subsubsection*{Note from the Arthur C Clarke story}

The restriction permits letters to repeat three times but not four may
seem surprising, this is noted in the story
\begin{quotation}
\lq{}[W]e use a special alphabet of our own. Modifying the
electromatic typewriters to deal with this is, of course, trivial. A
rather more interesting problem is that of devising suitable circuits
to eliminate ridiculous combinations. For example, no letter must
occur more than three times in succession.\rq{}

\lq{}Three? Surely you mean two.\rq{}

\lq{}Three is correct: I am afraid it would take too long to explain why, even if you understood our language.\rq{}
\end{quotation}

\end{document}
