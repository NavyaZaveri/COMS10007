%4_search.tex
%notes for the course PandA1 COMS10007 taught at the University of Bristol
%2018 Conor Houghton conor.houghton@bristol.ac.uk

%To the extent possible under law, the author has dedicated all copyright 
%and related and neighboring rights to these notes to the public domain 
%worldwide. These notes are distributed without any warranty. 


\documentclass[11pt,a4paper]{scrartcl}
\typearea{12}
\usepackage{graphicx}
\usepackage{pstricks}
\usepackage{listings}
\usepackage{color}
\lstset{language=C}
\usepackage{fancyhdr}
\pagestyle{fancy}
\lfoot{\texttt{github.com/conorhoughton/COMS10007}}
\lhead{COMS10007 - algorithms 4\_search (f) - Conor}
\begin{document}

\subsection*{4 Search: linear and binary search example}

Lets do a quick example of working out big oh: searching for the index
of an element in a sorted list. A completely terrible way to do this
is linear search, this is terrible because it doesn't make use of the
fact that the list is sorted. A code listing is given in
Table~\ref{c_linear_search}. We can see straight away that the code
between lines 6 and 10 is run $n$ times in the worst case, everything
else is run once and so this algorithm is $O(n)$.

\begin{table}
\begin{lstlisting}[numbers=left]
int search(int a[],int n, int val)
{

  int i;

  for(i=0;i<n;i++)
    {
      if(a[i]==val)
	return i;
    }

  return -1;
}
\end{lstlisting}
\caption{Linear search. This function searches the entries in the array a and returns the index when it finds val, if it doesn't find val it returns -1. The program {\tt linear\_search.c} implements this.\label{c_linear_search}.}
\end{table}

A much better way to search a sorted array is binary search. This is
an example of a \lq{}divide and conquer\rq{} algorithm, many of the
fastest algorithms use divide and conquer. It will be clear to you
that this algorithm would be better written using recursion, this is
typical of divide and conquer, but we haven't looked at analysing
recursion yet. The idea is to divide the array in half and check which
half, the half with bigger numbers or the half with smaller numbers,
the value we are searching for belongs to and to keep doing this,
dividing the remaining part of the array into two parts again and again
until the remaining part of the array that is being searched has only
one element. A code listing is given in Table~\ref{c_binary_search}.

\begin{table}
\begin{lstlisting}[numbers=left] 
int search(int a[],int n, int val)
{
  int mid, low=0, high=n-1;

  while(low<=high){
      mid=(low+high)/2;
      if(a[mid]==val)
	return mid;
      else if(val>a[mid])
	low=mid+1;
      else
	high=mid -1;
    }

  return -1;
}
\end{lstlisting}
\caption{Binary search. This function starts in the middle of the array
  and checks if the value there is bigger or smaller than val, if it
  is bigger then it does the same in the bottom half of the array, if it
  is smaller, in the bottom half and then repeats until there are no
  elements left. The program {\tt binary\_search.c} implements
  this.\label{c_binary_search}}
\end{table}

This search is extremely fast. There is a chance that a[mid]==val,
after a small number of iterations, indeed, if the middle value of the
array is the search value it will halt after only one
iteration. However, as usual, we assume the worst case, in which case
the algorithm runs to end, dividing the number of elements in half
each time. Ignoring the integer rounding effects, it goes like
Table~\ref{tab_binary_search}. Starting with $n$ states each
subsequent iteration halves the number of states until the last one
when there is one state left. Thus
\begin{equation}
1=\frac{n}{2^{T(n)-1}}
\end{equation}
and taking the log of both sides
\begin{equation}
0=\log_2{n}-(T(n)-1)
\end{equation}
using $\log_2{2}=1$, $\log_2{a^b}=b\log_2{a}$ and $\log_2{2}=1$. Hence
\begin{equation}
T(n)=\log_2{n}+1
\end{equation}
and this algorithm is $O(\log_2{n})$.

\begin{table}
\begin{center}
\begin{tabular}{llllllll}
1&2&3&4&\ldots&k&\ldots&T(n)\\
\hline
$n$&$\frac{n}{2}$&$\frac{n}{4}$&$\frac{n}{8}$&$\ldots$&$\frac{n}{2^{k-1}}$&$\ldots$&1
\end{tabular}
\end{center}
\caption{The number of elements left for binary search.\label{tab_binary_search}}
\end{table}


So, to reiterate; the usual way to examine the behavior of an
algorithm is to look at the worst case run time. This is because the
best case run time is often exceptional, like the one for binary
search if the first guess happens to be correct. The average run time
is often hard to calculate, both because it is often difficult to do
the mathematics and because it would often mean having some
description of how the initial data is distributed. Typically the
worst run time is also \lq{}of the same order\rq{} as the average run
time. We will see an exception to this later on in the case of quick
sort in which the worst case behavior is unusual. The big-Oh notation
is used for describing an algorithm, if the algorithm is said to be
$O(g(n))$ we mean $T(n)\in O(g(n))$ no matter what the initial
condition. Since $O(g(n))$ involves an upper bound $T(n)<cg(n)$ this
makes sense.
 
\end{document}
