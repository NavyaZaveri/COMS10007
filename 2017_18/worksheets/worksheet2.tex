%worksheet2.tex
%problem set for the course COMS10007 taught at the University of Bristol
%Conor Houghton conor.houghton@bristol.ac.uk

%To the extent possible under law, the author has dedicated all copyright 
%and related and neighboring rights to these notes to the public domain 
%worldwide. These notes are distributed without any warranty. 



\documentclass[11pt,a4paper]{scrartcl}
\typearea{12}
\usepackage{graphicx}
\usepackage{pstricks}
\usepackage{listings}
\usepackage{color}

\newif\ifanswers
\answerstrue

\lstset{language=C}
\pagestyle{headings}
\markright{COMS10007 - algorithms worksheet 2 - Conor}
\begin{document}

\subsection*{Algorithms Worksheet 2}

For each part of a question write the answer and include
workings. Questions are worth two marks each, there are also two marks
for attendance.

\begin{enumerate}

\item Solve
\begin{equation}
T(n)=-T(n-1)+4
\end{equation}
with $T(0)=1$. 

\ifanswers

\noindent Solution:

Substitute in $T(n)=A(-1)^n+B$ to get
\begin{equation}
A(-1)^n+B=-A(-1)^{n-1}-B+4
\end{equation}
so $2B=4$ or $B=2$; now $T(n)=A(-1)^n+2$ so $1=T(0)=A+2$ and hence $A=-1$.

\fi

\item Solve for $T(n)$ using the ansatz $T(n)=r^n$ for the following
  two step recursion relations. Solving for $r$ will give two values
  $r_1$ and $r_2$, this means that the general solution will be
  $T(n)=Ar_1^n+Br_2^n$. Use the two base values to find $A$ and $B$. 

\begin{enumerate}
\item $T(n)=2T(n-1)+3T(n-2)$ with $T(0)=0$ and $T(1)=4$.
\item $T(n)=T(n-2)$ with $T(0)=0$ and $T(1)=2$.
\end{enumerate}


\ifanswers 

\noindent Solution:

For (a) we have 
\begin{equation}
r^2=2r+3
\end{equation}
so $r^2-2r-3=0$ or $(r-3)(r+1)=0$ so
\begin{equation}
T(n)=3^nA+(-1)^nB
\end{equation}
and the initial conditions give $A+B=0$ and $3A-B=4$ so 
\begin{equation}
T(n)=3^n-(-1)^n
\end{equation}
For (b) we get $r^2=1$ so 
\begin{equation}
T(n)=A+(-1)^nB
\end{equation}
and the initial conditions give $A+B=0$ and $A-B=2$ so
\begin{equation}
T(n)=1-(-1)^n
\end{equation}
\fi

\item This question is about the master theorem. Use it to
  calculate big-Theta for $T(n)$ in each case. 

\begin{enumerate}
\item $T(n)= 25T(n/5)+4n^2$
\item $T(n)= 20T(n/5)+4n$
\item $T(n)= 16T(n/2)+2n^4$
\end{enumerate}

\ifanswers

\noindent Solution: for the first one $\log_5{25}=2$ and $c=2$ so
this is the middle case and $T(n)\in \Theta(n^2\log n)$, for the
second $\log_5 20>1$ so it is the first case and $T(n)\in
\Theta(n^{\log_5{20}})$; the last one is in the middle case as well
since $\log_2{16}=4$ and $T(n)\in \Theta(n^4\log{n})$.

\fi

\item Bubble sort $(3,5,2,8,4)$ showing each step.

\ifanswers
\noindent Solution:

\begin{tabular}{ccccc}
3&5&2&8&4\\
3&2&5&8&4\\
3&2&5&4&8\\
2&3&5&4&8\\
2&3&4&5&8
\end{tabular}
\end{enumerate}
\fi

\noindent Extra question to do at home if you want: write a recursive
version of quicksort to show that in the worst case
\begin{equation}
T(n)=T(n-1)+cn
\end{equation}

\noindent Solution:

\begin{lstlisting}[numbers=left]
void swap(int a[],int i, int j)
{
   int temp=a[i];
   a[i]=a[j];
   a[j]=temp;
}


void bubble(int a[], int n)
{

  if n=0 return;

  for(i=0;i<n-1;i++){
    if(a[i]>a[i+1]){
      swap(a,i,i+1);	    
    }
  }
  bubble(int a[], int n-1)
}
\end{lstlisting}

\end{document}
